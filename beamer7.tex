\documentclass{beamer}
\usetheme{CambridgeUS}

\title{Assignment 7 : Papoulis Textbook }
\author{Akshitha Kola}
\date{\today}
\logo{\large \LaTeX{}}

\usepackage{amsmath}
\setbeamertemplate{caption}[numbered]{}
\providecommand{\pr}[1]{\ensuremath{\Pr\left(#1\right)}}
\providecommand{\cbrak}[1]{\ensuremath{\left\{#1\right\}}}

\begin{document}

\begin{frame}
    \titlepage 
\end{frame}

\logo{}

\begin{frame}{Outline}
    \tableofcontents
\end{frame}

\section{Question}
\begin{frame}{Question}
    \begin{block}{Chapter 8 Example 8.28}
    We are given an N ($\eta$, 1) random variable x and we wish to test the simple hypothesis $\eta = \eta_{0}$ against $\eta \neq \eta_{0}$
    \end{block}
\end{frame}

\section{Solution}
\begin{frame}{Solution}
\frametitle{Solution}
In this problem $\eta_{m0} = \eta_{0}$ and 
\begin{align}
f(X,\eta) = \frac{1}{\sqrt{(2\pi)^{n}}}exp\left\lbrace -\frac{1}{2}\sum {(x_{i} - \eta)^{2}}\right\rbrace
\end{align}
The above expression is maximum if the sum 
\begin{align}
\sum {(x_{i} - \eta)^{2}} = \sum {(x_{i} - \bar{x})^{2}} + n(\bar{x} - \eta)^{2}
\end{align}
is minimum, that is $\eta = \bar{x}$. Hence $\eta_{m} = \bar{x}$.
\begin{align}
\lambda = \frac{exp\left\lbrace -\frac{1}{2}\sum {(x_{i} - \eta)^{2}}\right\rbrace}{exp\left\lbrace -\frac{1}{2}\sum {(x_{i} - \bar{x})^{2}}\right\rbrace} = exp\left\lbrace -\frac{n}{2}(\bar{x} - \eta_{0})^{2}\right\rbrace
\end{align}
\end{frame}

\begin{frame}
\begin{align}
 \textbf{w} = -2log\lambda = n(\bar{x} - \eta_{0})^{2} = (\bar{x} - \eta_{0})^{2} = \frac{(\bar{x} - \eta_{0})^{2}}{\frac{1}{\sqrt{n}}}
 \end{align}
 The right side is a random variable with $(\chi)^{2}$(1) distribution. Hence the random variable 
\textbf{w} has a $(\chi)^{2}(m - m_{0})$ distribution not only asymptotically, but for any n. 
\end{frame}
\end{document}
